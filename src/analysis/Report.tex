% Options for packages loaded elsewhere
\PassOptionsToPackage{unicode}{hyperref}
\PassOptionsToPackage{hyphens}{url}
%
\documentclass[
]{article}
\usepackage{amsmath,amssymb}
\usepackage{lmodern}
\usepackage{iftex}
\ifPDFTeX
  \usepackage[T1]{fontenc}
  \usepackage[utf8]{inputenc}
  \usepackage{textcomp} % provide euro and other symbols
\else % if luatex or xetex
  \usepackage{unicode-math}
  \defaultfontfeatures{Scale=MatchLowercase}
  \defaultfontfeatures[\rmfamily]{Ligatures=TeX,Scale=1}
\fi
% Use upquote if available, for straight quotes in verbatim environments
\IfFileExists{upquote.sty}{\usepackage{upquote}}{}
\IfFileExists{microtype.sty}{% use microtype if available
  \usepackage[]{microtype}
  \UseMicrotypeSet[protrusion]{basicmath} % disable protrusion for tt fonts
}{}
\makeatletter
\@ifundefined{KOMAClassName}{% if non-KOMA class
  \IfFileExists{parskip.sty}{%
    \usepackage{parskip}
  }{% else
    \setlength{\parindent}{0pt}
    \setlength{\parskip}{6pt plus 2pt minus 1pt}}
}{% if KOMA class
  \KOMAoptions{parskip=half}}
\makeatother
\usepackage{xcolor}
\usepackage[margin=1in]{geometry}
\usepackage{graphicx}
\makeatletter
\def\maxwidth{\ifdim\Gin@nat@width>\linewidth\linewidth\else\Gin@nat@width\fi}
\def\maxheight{\ifdim\Gin@nat@height>\textheight\textheight\else\Gin@nat@height\fi}
\makeatother
% Scale images if necessary, so that they will not overflow the page
% margins by default, and it is still possible to overwrite the defaults
% using explicit options in \includegraphics[width, height, ...]{}
\setkeys{Gin}{width=\maxwidth,height=\maxheight,keepaspectratio}
% Set default figure placement to htbp
\makeatletter
\def\fps@figure{htbp}
\makeatother
\setlength{\emergencystretch}{3em} % prevent overfull lines
\providecommand{\tightlist}{%
  \setlength{\itemsep}{0pt}\setlength{\parskip}{0pt}}
\setcounter{secnumdepth}{-\maxdimen} % remove section numbering
\ifLuaTeX
  \usepackage{selnolig}  % disable illegal ligatures
\fi
\IfFileExists{bookmark.sty}{\usepackage{bookmark}}{\usepackage{hyperref}}
\IfFileExists{xurl.sty}{\usepackage{xurl}}{} % add URL line breaks if available
\urlstyle{same} % disable monospaced font for URLs
\hypersetup{
  pdftitle={Airbnb New Year's Eve},
  pdfauthor={Fleur Le Mire, Mariëlla van Erve, Nishtha Staice, Yi Ting Tsai, Hilal Nur Turer},
  hidelinks,
  pdfcreator={LaTeX via pandoc}}

\title{Airbnb New Year's Eve}
\author{Fleur Le Mire, Mariëlla van Erve, Nishtha Staice, Yi Ting Tsai,
Hilal Nur Turer}
\date{maart 23, 2023}

\begin{document}
\maketitle

\textbf{\emph{Introduction}}

To answer our research questions, we used data from InsideAirbnb to
collect listings and calendar data from four major European cities:
London, Paris, Amsterdam, and Rome. After extracting the datasets for
each city, the data was cleaned by converting the missing values in
dummy variables to zero (0), removing the rows with missing values from
the remaining variables, and by excluding variables that are unrelated
to the analysis. Lastly, we merged the datasets from the individual
cities into one complete dataset.

This report provides a description of the variables used in the
analysis, including relevant summary statistics. Furthermore, the report
includes several plots to visualize the key variables from the analysis.
The main section of the report contains two analyses that answer our
research questions, with the results also discussed. Ultimately, we draw
a conclusion on the effect of New Year's Eve on Airbnb listings based on
these analyses.

The following questions will be answered:

\begin{enumerate}
\def\labelenumi{\arabic{enumi}.}
\item
  What is the effect of New Year's Eve on the price of Airbnb listings
  in capital cities in Europe?
\item
  Does New Year's Eve increase the likelihood of booking on Airbnb in
  capital cities in Europe?
\end{enumerate}

\textbf{\emph{Dependent Variable Descriptive}}

\textbf{Price}

In order to analyze the influence of New Year's Eve on the prices of
Airbnb accommodations, we used the metric variable `price'. In addition,
we created a dummy variable `newyearseve', which takes the value of 1 on
New Year's Eve and 0 on any other day. The average price of an Airbnb
listing from the whole data set is 159.0026453.

During analysis, We compare the price on New Year's Eve with the price
on days 5 days before and after the celebration. The results show a
P-value of less than 0.01 for both the complete model and the different
cities separate. Therefore, with a significance of 0.05, we can state
that there is a significant relationship between New Year's Eve and
prices of Airbnb listings. However, when we look into the outcome of
this comparison in the boxplot below, we can find that the difference
between the averages price of accommodations on New Year's Eve and
`usual' days is actually very small.

\includegraphics{../../gen/analysis/output/price_newyearseve_boxplot.pdf}

On the other hand, when we compare the cities using the boxplot below,
it can be concluded that the average price for Airbnb accommodations is
highest in Amsterdam on New Year's Eve, while the average price for
Airbnb accommodations is lowest in London.

\includegraphics{../../gen/analysis/output/price_per_city_boxplot.pdf}

\textbf{Booked}

To analyse the impact of New Year's Eve on the booking availability, we
created a dummy variable for booked accommodations. The dummy variable
``booked'' consists of two categories, where the value of 1 means that
an accommodation has been booked while the value of 0 means that an
accommodation has not been booked.

Looking at the mean 0.8100522 in the data set, we can infer that Airbnbs
were more frquently booked (than not booked) during the chosen period of
the data set (26th of December - 5th of January). This conclusion can be
corroborated by printing out the table.

Furthermore, 87.8\% of Airbnbs were booked in London, Paris, Amsterdam
and Rome on New Year's Eve, while only 80.3\% of the Airbnbs were booked
on usual days. Therefore, we can conclude that there are more
accommodations booked via Airbnb during New Year's Eve, but the
percentage is only slightly increased.

\textbf{\emph{Analysis}}

\textbf{Price analysis}

This analysis will answer the following question: ``What is the effect
of New Year's eve on the price of Airbnb listings?''

To address this question, we conducted a linear regression analysis
since we are exploring the relationship between a metric dependent
variable and a metric independent variables. In addition, we also want
to observe what is the effect of different cities, which are included as
dummy variables, on the relationship between New Year's Eve and price.

Foremost, we examined the linear regression for the main effect. The
result shows a P-value which is less than 0.05. With a significance of
0.05 we can state that there is a significant relationship between New
Year's Eve and prices of all Airbnb listings we studied. Moreover, we
ran the linear regression for the four cities separately and the results
also show P-values of less than 0.05 which represent a significant
relationship as well.

As below, we can also see that the average price in different cities is
higher on New Year's Eve compared to the usual days.

\includegraphics{../../gen/analysis/output/london_mean_price_graph.pdf}
\includegraphics{../../gen/analysis/output/paris_mean_price_graph.pdf}
\includegraphics{../../gen/analysis/output/ams_mean_price_graph.pdf}
\includegraphics{../../gen/analysis/output/rome_mean_price_graph.pdf}

Lastly, we included the interaction between New Year's Eve and the
different cities. The p-value is less than 0.05 when the cities effect
is included in the relationship. Therefore we can conclude that null
hypothesis can be rejected. The price will significantly be affected by
New Year's Eve in the cities we studied.

\textbf{Bookings analysis}

This analysis will answer the following question: ``Does New Year's Eve
increase the likelihood of Airbnb bookings in capital cities in
Europe?'' A logistic regression is performed here, as we set dependent
variable ``booked'' as a dummy variable.

Foremost, we performed a logistic regression for all cities combined.
The logistic regression analysis showed a P-value of less than 0.05.
Therefore, with a significance of 0.05, the H0 can be rejected. A
significant relationship between New Year's Eve and the likelihood of
booking an Airbnb cannot be rejected. When looking at the exponents of
the logistic regression, we can also conclude that on New Year's Eve,
the odds of a Airbnb being booked, increased with 1.77.

\begin{table}[h] \centering 
  \caption{Effect of New Years Eve on Number of Bookings of Airbnb Listings} 
  \label{} 
\begin{tabular}{@{\extracolsep{5pt}}lc} 
\\[-1.8ex]\hline 
\hline \\[-1.8ex] 
 & \multicolumn{1}{c}{Number of bookings} \\ 
\cline{2-2} 
\\[-1.8ex] &  \\ 
 & Total \\ 
\hline \\[-1.8ex] 
 New Year's Eve & 1.765$^{*}$ \\ 
  & (1.012) \\ 
  Constant & 4.084$^{***}$ \\ 
  & (1.003) \\ 
 \hline \\[-1.8ex] 
Observations & 839,236 \\ 
Log Likelihood & $-$406,609.300 \\ 
Akaike Inf. Crit. & 813,222.500 \\ 
\hline 
\hline \\[-1.8ex] 
\textit{Note:}  & \multicolumn{1}{r}{$^{*}$p$<$0.1; $^{**}$p$<$0.05; $^{***}$p$<$0.01} \\ 
\end{tabular} 
\end{table}

\textbf{\emph{Conclusion}}

Overall, to answer our research questions, both H0 have been rejected by
a p-value \textless{} 0.05. The significant relationship has been found
between New Year's Eve and the prices of Airbnb listings. In addition, a
significant relationship has also been found between New Year's Eve and
the chance of an Airbnb listing being booked.

\end{document}
